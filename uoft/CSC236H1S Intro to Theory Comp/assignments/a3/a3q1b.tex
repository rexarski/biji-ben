\documentclass{article}
\usepackage[utf8]{inputenc}
\usepackage{amssymb}
\usepackage{mathtools}
\usepackage{amsmath}
\usepackage{listings}

\title{Assignment 3}
\author{Rui Qiu (c3qiurui), Yilin Xie (c4xieyil), Yan Zeng (c3zengya)}
\date{March 26, 2015}
        

\begin{document}
\maketitle

Proof: Assume the DFA we created in part(a) is called $M$.

CLAIM:  No smaller DFA can compute the same language, i.e., $M$ is a minimal DFA for $L(M)$.

SUPPOSE: For the sake of contradiction, suppose that there exists a smaller DFA $M'=\{Q', \Sigma', \delta', s_0', F'\}$ for $L$ such that $|Q'| \leqslant 2$.

Consider the following strings:

\begin{center}
$x_1 = 0$

$x_2 = 1$

$x_3 = 2$

$x_4 = 3$

$x_5 = 4$

$x_6 = 5$

$x_7 = 6$

$x_8 = 7$

$x_9 = 8$

$x_{10} = 9$
\end{center}

These strings are chosen such that the computation of these strings takes us into each of the three states in $M$. Observe this:

\begin{center}
$\hat{\delta}(s_0, 0) = s_0$

$\hat{\delta}(s_0, 1) = s_1$

$\hat{\delta}(s_0, 2) = s_2$

$\hat{\delta}(s_0, 3) = s_0$

$\hat{\delta}(s_0, 4) = s_1$

$\hat{\delta}(s_0, 5) = s_2$

$\hat{\delta}(s_0, 6) = s_0$

$\hat{\delta}(s_0, 7) = s_1$

$\hat{\delta}(s_0, 8) = s_2$

$\hat{\delta}(s_0, 9) = s_0$
\end{center}

By specifying $s_0, s_1, s_2$, we can reduce the number of cases into 3 major cases:
\begin{enumerate}
\item $s_0$: The string $w$ is a multiple of 3.
\item $s_1$: The string $w$ is 1 modulo 3.
\item $s_2$: The string $w$ is 2 modulo 3.
\end{enumerate}

Therefore, the simplified version is:

\begin{center}
$\hat{\delta}(s_0, t_0) = s_0$

$\hat{\delta}(s_0, t_1) = s_1$

$\hat{\delta}(s_0, t_2) = s_2$

where $t_0 \equiv 0 \pmod 3, t_1 \equiv 1 \pmod 3, t_2 \equiv 2 \pmod 3, t_i \in \{0,1,2,...,9\}$.
\end{center}

By the pigeonhole principle, two of these computations $\hat{\delta}$ on strings $x_1$ to $x_3$ must yield the same state in $M'$ since the number of states is smaller than 3. Therefore, we must show for each pair of computations $(\hat{\delta}(s_0,t_i),\hat{\delta}(s_0,t_j))$ that:

\begin{center}
$\hat{\delta}(s_0,t_i) \not = \hat{\delta}(s_0,t_j)$, where $i, j \in \{0, 1, 2\}, i\not = j$.
\end{center}

There are ${3 \choose 2}$ cases we must show contradict our assumption. One of the cases is illustrated below:

CASE 1: Show contradiction for $t_0$ and $t_1$. We start with the following statement:

\begin{center}
$\hat{\delta}(s_0,t_0)  = \hat{\delta}(s_0,t_1)$
\end{center}

Note that by the defintion of $\hat{\delta}$ we can add the same string to both sides without affecting the equality. For example,  we add 3 to both sides and get::

\begin{center}
$\hat{\delta}(s_0,t_03)  = \hat{\delta}(s_0,t_13)$
\end{center}

However, our language should accept the string $t_03$ but not the string $t_13$, since the latter is not a multiple of 3. Therefore $\hat{\delta}(s_0,t_03) \in F $ but $\hat{\delta}(s_0,t_13) \not \in F$. Thus these two computations can not result in the same state, giving us a contradiction.

By proving each of the cases results in a contradiction, we prove that our DFA is indeed minimal, i.e., there is no smaller DFA can compute the language.

\end{document}