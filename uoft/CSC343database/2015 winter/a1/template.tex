\documentclass{article}
\usepackage{fullpage}
\usepackage[normalem]{ulem}
\usepackage{amstext}
\newcommand{\var}[1]{\mathit{#1}}
\setlength{\parskip}{6pt}

\begin{document}

~~~\vspace{-2.0cm}

\noindent
University of Toronto\\
{\sc csc}343, Winter 2015\\[10pt]
{\LARGE\bf Assignment 1: Your name and student number here}

\noindent
Unary operators on relations:
\begin{itemize}
\item $\Pi_{x, y, z} (R)$
\item $\sigma_{condition} (R) $
\item $\rho_{New} (R) $
\item $\rho_{New(a, b, c)} (R) $
\end{itemize}
Binary operators on relations:
\begin{itemize}
\item $R \times S$
\item $R \bowtie S$
\item $R \bowtie_{condition} S$
\item $R \cup S$
\item $R \cap S$
\item $R - S$
\end{itemize}
Logical operators:
\begin{itemize}
\item $\vee$
\item $\wedge$
\item $\neg$
\end{itemize}
Assignment:
\begin{itemize}
\item $New(a, b, c) := R$
\end{itemize}

\noindent
Below is the text of the assignment questions; we suggest you include it in your solution.
We have also included a nonsense example of how a query might look in LaTeX.  
We used \verb|\var| in a couple of places to show what that looks like.  
If you leave it out, most of the time the algebra looks okay, but names
such as ``Offer" look horrific without it.

The characters ``\verb|\\|" create a line break and ``[5pt]" puts in 
five points of extra vertical space.  The algebra is easier to read with extra
vertical space.
We chose ``---" to indicate comments, and added less vertical space between comments
and the algebra they pertain to than between steps in the algebra.
This helps the comments visually stick to the algebra.

%----------------------------------------------------------------------------------------------------------------------
\section*{Part 1: Queries}

\begin{enumerate}

\item   % ----------
Report the user name of every student who has never worked with anyone, but
has indeed submitted at least one file for at least one assignment.

{\bf Answer:}\\[5pt]
{\large
--- IDs of all big Floops:  \\[5pt]
$
BigFloop(ID) := 
\Pi_{One.who} \sigma_{One.who = \var{Offer.LID} \wedge One.what > 10} (One \times \var{Offer}) \\[10pt]
$
--- Name and favourite ice cream of those Floops: \\[5pt]
$
\Pi_{name, flavour} Floop \bowtie Everyone
$
}

\item   % ----------
Find the graders who have marked every assignment.
We will say that a grader has marked an assignment if
they have given a grade on that assignment to at least one group,
whether or not that grade has been released.
Report the grader's userName.

\item   % ----------
Find all groups for A2
({\it i.e.}, the assignment whose description is ``A2")
whose last submission was after the due date, but who submitted at least two different files
({\it i.e.}, files with two different names) before the due date.
Report the group ID, the name of the first file they submitted, and when they submitted it.
If there are ties for a group's first submit, report them all.

\item   % ----------
Find pairs of students who worked in a group together, and without any other students,
on each assignment in the database that allowed groups of size two or more.
Report their user names, last names, and firstnames.

\item   % ----------
Find any assignments where 
the highest mark given by one grader is less than 
the lowest mark given by another grader.
In your result, include a row for each grader on each of these assignments.
Report the assignment ID, the grader's userName, and their minimum and maximum grade.

\item   % ----------
Find all students who have worked in a group with at least one other person, 
but have never worked with the same person
twice.
Report their userName.

\item   % ----------
Find all students who meet these two requirements:
(a) their groups (whether they were working alone or with others)
handed in every required file, and did so on time, for all assignments, and
(b) their grades never went down from one assignment to another with a later due date.
Report their userName.\\
Note: If an assignment has no required files, it is true of any group that they handed in every required file.

\item   % ----------
Find all assignments that have one or more groups with no grade or with a grade that has not
been released.
Report the assignment ID and description.

\item   % ----------
Assignments may have required files, but students can also hand in other files that are not required.
Find all groups that never handed in a file that was not required.
Report the group ID.

\end{enumerate}



%----------------------------------------------------------------------------------------------------------------------
\section*{Part 2: Additional Integrity Constraints}

Express the following integrity constraints
with the notation $R = \emptyset$, where $R$ is an expression of relational algebra. 
You are welcome to define intermediate results with assignment
and then use them in an integrity constraint.

\begin{enumerate}

\item   % ----------
No grades can be released for an assignment 
unless every group has been given a grade on that assignment (whether or not it has been released).

\item   % ----------
A TA can't give a grade to any groups on an assignment unless they have completed marking 
({\it i.e.,} they have given a grade, whether or not it has been released)
for every group they were assigned to grade on every assignment with an earlier due date.

\item   % ----------
A TA can't be assigned to grade a group of size two or more unless he or she has already 
given a grade (that has been released) to at least 3 students (each working in a group of size 1) 
on an assignment with an earlier due date.

\end{enumerate}

\end{document}



