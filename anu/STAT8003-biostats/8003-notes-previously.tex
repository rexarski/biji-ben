\documentclass[a4paper, 11pt, twoside]{article}
\usepackage{amssymb}
\usepackage{amsmath}
\begin{document}
\title{Golden Braids of Biostats}
\author{Rui Qiu}
\date{2018-03-05}

\maketitle

\section{Life Table}
$l_x$: number of persons alive at age $x$.\\
$d_x$: number of persons dying between ages $x$ and $x+1$.\\
$p_x$: prob a person aged $x$ survives to age $x+1$.\\
$q_x$: prob a person aged $x$ dies before age $x+1$.\\
${}_tp_x$: prob a person aged $x$ survives to age $x+t$.\\
${}_tq_x$: prob a person aged $x$ dies before age $x+t$.\\
\\\textbf{Mutual expressions:}
\[\begin{split}
	{}_tp_x&=\frac{l_{x+t}}{l_x}\\
	d_x &= l_x-l_{x+1}\\
	d_{x+1} &=l_{x+1}-l_{x+2}\\
	&\vdots\\
	d_{x+T}&=l_{x+T}-l_{x+T+1}=0\\
	l_x&=d_x+d_{x+1}+\cdots +d_{x+T}=\sum^T_{t=0}d_{x+T}\\
	q_x&=1-p_x=\frac{d_x}{l_x}\\
	{}_tq_x&=1-{}_tp_x=\frac{l_x-l_{x+t}}{l_x}
\end{split}\]

\section{The Force of Mortality}
$\mu_x$ is the \textbf{instantaneous} rate of mortality at age $x$, as an annualized rate ($l_x$ is continuous and differentiable)

\[\mu_x=-\frac1{l_x}\frac{d}{dx}l_x=-\frac{d}{dx}\log(l_x)\]

By definition of derivative,

\[\begin{split}
\mu_x &= -\frac1{l_x}\frac{dl_x}{dx}\\
&=-\frac1{l_x}\lim_{h\to 0}\frac{l_{x+h}-l_x}{h}\\
&\approx -\frac1{l_x}\cdot\frac{l_{x+h}-l_x}{h}\\
&=\frac{1}{h}\frac{l_x-l_{x+h}}{l_x}\\
&=\frac1h(1-{}_hp_x)\\
&=\frac1h{}_hq_x\\
&=\frac{{}_hq_x}{h}	
\end{split}
\]

\textbf{Use force of mortality to compute survival time}
\[\begin{split}
\mu_{x+s}&=-\frac{d\ln l_{x+s}}{ds}\\
\implies \int^n_0\mu_{x+s}ds&=\int^n_0-\frac{d\ln l_{x+s}}{ds}ds\\
&=[-\ln l_{x+s}]\Bigg\rvert^n_0\\
&=-\ln l_{x+n}+\ln l_x\\
&=-\ln\frac{l_{x+s}}{l_x}=-\ln({}_np_x)\\
\implies {}_np_x&=\exp\left(-\int^n_0\mu_{x+s}ds\right)	
\end{split}
\]

\textbf{Use force of mortality to compute ${}_nq_x$}

\[\begin{split}
	\mu_{x+s}&= -\frac{dl_{x+s}}{ds}\frac{1}{l_{x+s}}\\
	\frac{dl_{x+s}}{ds}&=-l_{x+s}\mu_{x+s}\\
	\int^n_0\frac{dl_{x+s}}{ds}ds&=-\int^n_0 l_{x+s}\mu_{x+s}ds\\
	&=\left[l_{x+s}\right]\Bigg\rvert^n_0=l_{x+n}-l_x\\
	&=-\int^n_0l_{x+s}\mu_{x+s}ds\\
	-\frac{l_{x+n}-l_x}{l_x}&=\int^n_0\frac{l_{x+s}}{l_x}\mu_{x+s}ds\\
	{}_nq_x&=\int^n_0{}_sp_x\mu_{x+s}ds\\
	l_x-l_{x+1}&=d_x=\int^1_0l_{x+s}\mu_{x+s}ds \text{ when $n=1$.}
\end{split}
\]

\textbf{Stats notation}\\

$T$ is survival time, the distribution of $T$ can be characterized by 3 equivalent quantities:

1. survival function $S(t)=P(T>t)$, but no censored observations in reality, so we use $\hat{S}(t)=\frac{\text{number of persons surviving longer than }t}{\text{total number of persons}}$

2.p.d.f. $f(t)=\frac{\lim_{\Delta t\to 0}P(\text{life dies in }(t,t+\Delta t)}{\Delta t}$

3.hazard function $\mu(t) \equiv \mu_t$:

\[\mu(t)=\frac{\lim_{\Delta t\to 0}P(\text{life dies in }(t,t+\Delta t),\text{ given alive at }t)}{\Delta t}\]\\

\textbf{Notation conversions}\\

\[\begin{split}
	{}_tq_x&=\frac{l_x-l_{x+t}}{l_x}=P(T_x\leq t)=F_{T_x}(t)\\
	{}_tp_x&=\frac{l_{x+t}}{l_x}=P(T_x\geq t)=S_{T_x}(t)\\
	\mu_{x+t}&=-\frac1{l_{x+t}}\frac{dl_{x+t}}{dt}=\mu(t)=\frac{f(t)}{S(t)}
\end{split}
\]

Note the difference between $f(t)$ and $\mu(t)$, where the hazard function is a conditional rate, but $f(t)$ is an unconditional failure rate.

\[
\begin{split}
	f(t)=\frac{dF(t)}{dt}&=\lim_{\Delta t\to 0}\frac{F(t+\Delta t)-F(t)}{\Delta t}\\
	&=\lim_{\Delta t\to 0}\frac{P(\text{dies in } (t,t+\Delta t))}{\Delta t}\\
	\mu(t)=\frac{f(t)}{S(t)}&=\lim_{\Delta t\to 0}\left(\frac{P(\text{dies in }(t,\Delta t+t))}{\Delta t\cdot S(t)}\right)\\
	&=\lim_{\Delta t\to 0}\left(\frac{\frac{P(\text{dies in }(t,t+\Delta t))}{P(\text{survives longer than }t)}}{\Delta t}\right)\\
	&=\lim_{\Delta t\to 0}\frac{P(\text{dies in }(t, t+\Delta t) \mid \text{survives longer than }t)}{\Delta t}
\end{split}
\]

\section{Force of Mortality (continued)}
We know, from last lecture, that the force of mortality (also known as the hazard function), can be expressed in terms of the survival function $S(t)$ and the p.d.f. $f(t)$:

\[\mu(t)=\mu_T(t)=\lambda_T(t)=\frac{f(t)}{S(t)}=\frac{F'(t)}{S(t)}=\frac{-(1-F(t))'}{S(t)}=\frac{-S'(t)}{S(t)}\]

where $S(t)$ is the prob alive at time $t$.

\paragraph{Example:} $T$ follows an exponential distribution with $f(t)=\lambda\exp(-\lambda t)$ express $S(t)$ in terms of $\mu(t)$.

\[
\begin{split}
	\mu(t)&=-\frac{S'(t)}{S(t)}\\
	\implies\frac{d}{dt}\log(S(t))&=-\mu(t)\\
	\implies \left[\log(S(r))\right]^t_0&=-\int^t_0\mu(r)dr\\
	\log\frac{S(t)}{1}&=-\int^t_0\mu(r)dr\\
	S(t)&=\exp\left(-\int^t_0\mu(r)dr\right)\\
	\mu(t)&=\frac{f(t)}{S(t)}=\frac{\lambda\exp(-\lambda t)}{\exp(-\lambda t)}=\lambda\\&\text{a constant force of mortality, which is unreal in real life}
\end{split}
\]

\textbf{Some "Laws" of Mortality}

\paragraph{Gompertz Law:} $\mu(t)=B\times C^t\implies \frac{\mu(t+1)}{\mu(t)}=C$ constant percentage increase. One of the way to estimate the parameter $B$ and $C$ is:

\[\log(\mu(t))=\log(B)+t\log(C)\]

If Gompertz law is appropriate, should have a linear relationship between $\log(\mu(t))$ and $t$.

\paragraph{Makehams Law:} $\mu(t)=A+B\times C^t$ adds the fixed component $A$ causes of death due to chance.

But note that both Gompertz and Makehams only work over relatively small age ranges.

\section{Complete vs Curtate Expected Future Lifetime}

A standard definition:

\[e^o_x=E[T_x]=\int^\infty_0tf(t)dt\]

is the \textbf{complete expected future lifetime} of person aged $x$. Another derivation of $e^o_x$ using ${}_tp_x$ is below:

\[
\begin{split}
	f(t)&={}_tp_x\mu(x+t)\\
	{}_nq_x&=\int^n_0{}_tp_x\mu(x+t)dt\\
	&=F_{T_x}(n)\\
	&=\int^n_0f(t)dt\ \ \ \ \text{prob dies in $n$ years}\\
	\mu(x+t)&=\frac{1}{l_{x+t}}\cdot\frac{dl_{x+t}}{dt}=\frac{-\frac{dl_{x+t}}{l_x}/dt}{\frac{l_{x+t}}{l_x}}\\
	&=\frac{-d_tp_x/dt}{{}_tp_x}\\
	\therefore\ e^o_x&=\int^\infty_0tf(t)dt=\int^\infty_0t\cdot_tp_x\frac{-d_tp_x/dt}{{}_tp_x}dt=\int^\infty_0t\left(\frac{-d_tp_x	}{dt}\right)dt,\ \ \text{ integrate by parts}\\
	&=\int^\infty_0{}_tp_xdt
\end{split}
\]

A rather non-standard expected future lifetime (\textbf{curtate expected future lifetime}):

\[e_x=E[K_x]=\sum^\infty_{k=0}kP(K_x=k)\]

where $K_x$ is the random variable representing the whole number of future years lived by a person aged $x$, i.e. $K_{20}=\lfloor T_{20}\rfloor$.\\

\textbf{Derivation of }$P(K_x=k)=_kp_x\times q_{x+k}$ and $e_x=\sum^\infty_{k=1}{}_kp_x$.

\[
\begin{split}
	e_x&=\sum^\infty_{k=0}k\cdot P(K_x=k)\\
	&=\sum^\infty_{k=0}k\cdot_kp_x\cdot q_{x+k}\\
	&=1\cdot p_x\cdot q_{x+1} + {}_2p_xq_{x+2} + {}_3p_xq_{x+3}+\cdots\\
	&+{}_2p_xq_{x+2}+{}_3p_xq_{x+3}+\cdots\\
	&+{}_3p_xq_{x+3}+\cdots\\
	&=\sum^\infty_{j=1}{}_jp_xq_{x+j}+\sum^\infty_{j=2}{}_jp_xq_{x+j}+\cdots\\
	&=\sum^\infty_{k=1}\sum^\infty_{j=k}{}_jp_xq_{x+j}\\
	&=\sum^\infty_{k=1}{}_kp_x
\end{split}
\]

with this:

\[
\begin{split}
	\sum^\infty_{j=k}{}_jp_xq_{x+j}&=\sum^\infty_{j=k}P(j\leq T_x < j+1)\\
	&=P(k\leq T_x<k+1)+P(k+1\leq T_x < k+2)\\
	&=P(T_x\geq k)\\
	&={}_kp_x
\end{split}
\]

Therefore, done.

\section{miscellaneous related to force of mortality}

\paragraph{Definition:} The probability that a life aged $x$ will survive for $n$ years but die during the subsequent $m$ years is called the deferred probability:

\[{}_{n\mid m}q_x=P(n<T_x<n+m)=\frac{l_{x+n}-l_{x+n+m}}{l_x}\]

For non-integer ages, we usually make one of the following assumptions about the trend in mortality rate between two whole ages:

1. Method One Uniform Distribution of Deaths (UDD): ${}_tp_x\mu_{x+t}=f(t)$ is a constant for $0<t<1$.

2. Method Two Constant Force of Mortality: $\mu_{x+t}=\mu=$ constant for $0<t<1$.

\paragraph{Definition:} The central rate of mortality ($m_x$) is an alternative to the initial rate of mortality $q_x$ with

\[m_x=\frac{d_x}{\int^1_0l_{x+t}dt}\]

more appropriate when we have data on the number of people aged between $x$ and $x+1$.

\section{Parametric estimation with Two methods}

\subsection{Method of Moments (MOM)}

The population moments for a random variable $X$ are $E(X^j)$ and the corresponding sample moments are $n^{-1}\sum^n_{i=1}x_i^j$.

The standard way to find more than one estimated parameters is: For $n$ parameters we need $n$ equations, i.e. up to $n$-th sample moment.


\end{document}