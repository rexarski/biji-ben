\documentclass[12pt]{article}

\usepackage{fancyhdr}
\usepackage{extramarks}
\usepackage{amsmath}
\usepackage{amsthm}
\usepackage{amsfonts}
\usepackage{tikz}
\usepackage[plain]{algorithm}
\usepackage{algpseudocode}

\usetikzlibrary{automata,positioning}

%
% Basic Document Settings
%

\topmargin=-0.45in
\evensidemargin=0in
\oddsidemargin=0in
\textwidth=6.5in
\textheight=9.0in
\headsep=0.25in

\linespread{1.1}

\pagestyle{fancy}
\lhead{\hmwkAuthorName}
\chead{\hmwkClass\ (\hmwkClassInstructor): \hmwkTitle}
\rhead{\firstxmark}
\lfoot{\lastxmark}
\cfoot{\thepage}

\renewcommand\headrulewidth{0.4pt}
\renewcommand\footrulewidth{0.4pt}

\setlength\parindent{0pt}

%
% Create Problem Sections
%

\newcommand{\enterProblemHeader}[1]{
    \nobreak\extramarks{}{Problem \arabic{#1} continued on next page\ldots}\nobreak{}
    \nobreak\extramarks{Problem \arabic{#1} (continued)}{Problem \arabic{#1} continued on next page\ldots}\nobreak{}
}

\newcommand{\exitProblemHeader}[1]{
    \nobreak\extramarks{Problem \arabic{#1} (continued)}{Problem \arabic{#1} continued on next page\ldots}\nobreak{}
    \stepcounter{#1}
    \nobreak\extramarks{Problem \arabic{#1}}{}\nobreak{}
}

\setcounter{secnumdepth}{0}
\newcounter{partCounter}
\newcounter{homeworkProblemCounter}
\setcounter{homeworkProblemCounter}{1}
\nobreak\extramarks{Problem \arabic{homeworkProblemCounter}}{}\nobreak{}

%
% Homework Problem Environment
%
% This environment takes an optional argument. When given, it will adjust the
% problem counter. This is useful for when the problems given for your
% assignment aren't sequential. See the last 3 problems of this template for an
% example.
%
\newenvironment{homeworkProblem}[1][-1]{
    \ifnum#1>0
        \setcounter{homeworkProblemCounter}{#1}
    \fi
    \section{Problem \arabic{homeworkProblemCounter}}
    \setcounter{partCounter}{1}
    \enterProblemHeader{homeworkProblemCounter}
}{
    \exitProblemHeader{homeworkProblemCounter}
}

%
% Homework Details
%   - Title
%   - Date
%   - Class
%   - Instructor
%   - Author
%

\newcommand{\hmwkTitle}{Homework\ \#1}
\newcommand{\hmwkDate}{February 24, 2017}
\newcommand{\hmwkClass}{MATH6222}
\newcommand{\hmwkClassInstructor}{Dr. David Smyth}
\newcommand{\hmwkAuthorName}{\textbf{Rui Qiu u6139152}}

%
% Title Page
%

\title{
    \vspace{2in}
    \textmd{\textbf{\hmwkClass:\ \hmwkTitle}}\\
    \normalsize\vspace{0.1in}\small{\hmwkDate}\\
    \vspace{0.1in}\large{\textit{\hmwkClassInstructor}}
    \vspace{3in}
}

\author{\hmwkAuthorName}
\date{}

\renewcommand{\part}[1]{\textbf{\large Part \Alph{partCounter}}\stepcounter{partCounter}\\}

%
% Various Helper Commands
%

% Useful for algorithms
\newcommand{\alg}[1]{\textsc{\bfseries \footnotesize #1}}

% For derivatives
\newcommand{\deriv}[1]{\frac{\mathrm{d}}{\mathrm{d}x} (#1)}

% For partial derivatives
\newcommand{\pderiv}[2]{\frac{\partial}{\partial #1} (#2)}

% Integral dx
\newcommand{\dx}{\mathrm{d}x}

% Alias for the Solution section header
\newcommand{\solution}{\textbf{\large Solution}}

% Probability commands: Expectation, Variance, Covariance, Bias
\newcommand{\E}{\mathrm{E}}
\newcommand{\Var}{\mathrm{Var}}
\newcommand{\Cov}{\mathrm{Cov}}
\newcommand{\Bias}{\mathrm{Bias}}

\begin{document}

\maketitle

\pagebreak

\begin{homeworkProblem}
	Prove that $\sqrt{11}$ is irrational. You may use the fact that every integer can be uniquely decomposed as a product of primes.
	\\
	
    \textbf{Proof:}
    
    The idea is very similar to the one we used to prove the irrationality of $\sqrt{2}$ in class.
    \\
    
    Suppose $\sqrt{11}$ is a rational number, i.e., for two co prime integers $p, q$, i.e.$p,q$ have no common factors, it can be written as
    
    \[
        \sqrt{11} = \frac{p}{q}
    \]
    
    Square the both sides and multiple by $q^2$ we have
    
    \[
    	11q^2 = p^2
    \]
    
    	
    Now we recall that some fact proved in class:
    
    \begin{itemize}
    	\item The product of two odd numbers is odd.
    	\item The product of two even numbers is even.
    	\item The product of an even and an odd is even.
    \end{itemize}
    
    and consider this:
    
    \begin{itemize}
    	\item If $p$ is odd, $q$ is even. Then the right hand side (RHS) is odd, the left hand side (LHS) is even. Contradiction.
    	\item If $p$ is even, $q$ is odd. Then RHS is even, LHS is odd. Contradiction.
    	\item If $p, q$ both even, contradicts the fact that $p, q$ are co primes.
    \end{itemize}
    
    Therefore, $p, q$ can only be two odd co primes.
    
    \[
    	\begin{split}
    		p &= 2n + 1 \\
    		q &= 2m + 1 \\ 
    		11(2m+1)^2 &= (2n+1)^2 \\
    		11(4m^2+4m+1) &= 4n^2+4n+1 \\
    		44m^2+44m+11 &= 4n^2+4n+1 \\
    		44m^2+44m+10 &= 4n^2+4n \\
    		22m^2+22m+5 &=2n^2+2n\\
    	\end{split}
    \]
    
    $22m^2, 22m, 2n^2, 2n$ are even numbers. So the RHS is even. But an even number $22m^2+22m$ plus an odd number $5$ equals an odd number (LHS). So we have a contradiction here.
    \\
    
    Hence, the original hypothesis that $\sqrt{11}$ is rational fails. So we proved that $\sqrt{11}$ is irrational.

\end{homeworkProblem}

\begin{homeworkProblem}
    Let $S$ denote the set of all prime numbers of the form $4k+3$ with $k \in \mathbb{N}.$ (So $3 \in S, 7 \in S,$ but $5\not\in S$). Prove that $S$ is infinite.
    \\
    
    \textbf{Proof}
    
    Suppose there are only finitely many primes $p_1, \dots p_k$ in the set $S$. Consider the number $N=4p_1\cdot p_2 \cdots p_k - 1 = 4(\Pi_{i=1}^{k}p_i-1)+3$ which is also of the form $4n+3$.
    \\

    Since it is greater than any $p_i$, so consider it not a prime. Then $N$ is divisible by a prime.
    \\
    
    Note that all integers should be one of the form $4n, 4n+1, 4n+2, 4n+3$. The factors of $N$ cannot be of the form $4n, 4n+2$ since $N$ is odd. On the other hand, none of the elements of $S$ divides $N$. So the only possible form of factors of $N$ is $4n+1$.
    \\
    
    However
    
    \[
    	\forall a, b \in \mathbb{Z}, (4a+1)(4b+1)=16ab+4a+4b+1=4(4ab+a+b)+1.
    \]

    So the product of any two primes of the form $4n+1$ is still $4n+1$. It's like an infinite loop. But remember that $N$ itself is of the form $4n+3$ in the end. Contradiction!
    \\
    
    Hence the original hypothesis is incorrect, i.e. $S$ is infinite.

\end{homeworkProblem}

%
% Non sequential homework problems
%

% Jump to problem 18
\begin{homeworkProblem}[4]
    Let $f$ and $g$ denote functions from $\mathbb{R}$ to $\mathbb{R}$. Recall that such a function is \textit{bounded} if there exists a real number $M$ such that $|f(x)| < M$ for all $x \in \mathbb{R}$. Determine whether each of the following statements are true. If true, provide a proof. If false, provide a counterexample.
    
    \begin{itemize}
    	\item If $f$ and $g$ are bounded, then $f+g$ is bounded.
    	\item If $f$ and $g$ are bounded, then $fg$ is bounded.
    	\item If $f+g$ is bounded, then $f$ and $g$ are bounded.
    	\item If $fg$ is bounded, then $f$ and $g$ are bounded.
    	\item If $f+g$ and $fg$ are bounded, then $f$ and $g$ are bounded.
    \end{itemize}
    
    You may use the \textit{triangle inequality} which states that for all $x,y \in \mathbb{R}$,
    
    \[ |x+y| \leq |x| + |y|. \]
    
    
    \textbf{Solution}
    \\
    
    \textbf{Statement 1} 
    \\
    
    True. According to the definition of \textit{boundedness}, $\forall x \in \mathbb{R}, \exists M_1, M_2 \in \mathbb{R}$ such that
    
    \[
    	\begin{split}
    		|f(x)| &< M_1 \\
    		|g(x)| &< M_2 \\
    		|(f+g)(x)| = |f(x)+g(x)| &\leq |f(X)| + |g(x)| < M_1+ M_2 = M
    	\end{split}
    \]
    
    Hence $f+g$ is bounded by $M=M_1+M_2$.
    \\
    
    \textbf{Statement 2}
    \\
    
    True. Similarly,
    
    \[
    	|fg(x)| = |f(x)g(x)| = |f(x)||g(x)| < M_1\cdot M_2 = M'
    \]
    
    Hence $fg$ is bounded by $M'=M_1\cdot M_2$.
    \\
    
    \textbf{Statement 3}
    \\
    
    False. Suppose $f(x) = \pi\cdot x, g(x) = -\pi\cdot x$, then $(fg)(x)=0$ which is bounded since $0 <= 0$ all the time. But neither of $f(x), g(x)$ is bounded.
    \\
    
    \textbf{Statement 4}
    \\
    
    False. The counterexample is similar to the one above. Suppose
    
    \[
    	\begin{split}
			f(x)&= 
			\left\{ \
			\begin{aligned}
				\frac{1}{x},\ &\forall x \in \mathbb{R}-\{0\} \\
				0,\ &x \in \{0\}
			\end{aligned} \right.
			\\
			g(x) &= x,\ \forall x \in \mathbb{R}. 
		\end{split} 
    \]
    
    The product of them, $fg(x) = 1$ is bounded, but neither $f$ nor $g$ is bounded.
    \\
    
    \textbf{Statement 5}
    \\
    
    True. Since $f+g$ and $fg$ are bounded, $\exists M_1, M_2 \in \mathbb{R}$, such that 
    
    \[
    	\begin{split}
    		|f(x)+g(x)| &< M_1	 \\
    		|f(x)\cdot g(x)| &< M_2 \\
    		|f(x)^2 + g(x)^2| &= |\left(f(x)+g(x)\right)^2 - 2f(x)g(x)|\\
    		&\leq |\left(f(x)+g(x)\right)^2|+2|f(x)g(x)|\\
    		&< M_1^2 + 2M_2
    	\end{split}
    \]
    
    This is to say, the sum of two squares $f^2+g^2$ is bounded. As we know, the magic of a square number is that it is always nonnegative. So
    
    \[
    	\begin{split}
    		f(x)^2 &\leq f(x)^2 + g(x)^2 = M_1^2 + 2M_2 \\
    		g(x)^2 &\leq f(x)^2 + g(x)^2 = M_1^2 + 2M_2 \\
    	\end{split}
	\]
	
	Hence $f(x) \leq \sqrt{M_1^2+2M_2},\ g(x) \leq \sqrt{M_1^2+2M_2},\ $i.e., $f$ and $g$ are both bounded.

\end{homeworkProblem}

\end{document}
