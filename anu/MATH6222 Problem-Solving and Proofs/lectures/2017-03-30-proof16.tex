\documentclass[a4paper, 11pt, twoside]{article}
\usepackage{amssymb}
\usepackage{amsmath}
\begin{document}
\title{MATH6222 week 6 lecture 16}
\author{Rui Qiu}
\date{2017-03-30}

\maketitle

Given integers $a,b,c$, claim

\[ax+by=c\]

has a solution in integers if and only if $\gcd(a,b)|c$.\\

\textbf{Proof:}

$(\Rightarrow)$ Suppose $ax+by=c$ has a solution in integers, i.e. $\exists\ m,n \in \mathbb{Z}$ such that $am+bn=c.$

By definition, $\gcd(a,b)|a$ and $\gcd(a,b)|b$.

By properties of divisibility, $\gcd(a,b)|am+bn\implies \gcd(a,b)|c$, we are done.

$(\Leftarrow)$ Suppose $\gcd(a,b)|c$, then I must show $\exists\ m,n\in\mathbb{Z}$ such that $am+bn=c$.

First, suppose we already have $mn,n$ such that $am+bn=\gcd(a,b)$.

Then I can get any other multiple of $\gcd(a,b)$ as follows:

If $c=k\gcd(a,b)$,

Let $m'=km, n'=kn$, then $am'+bn'=k(am+bn)=k\gcd(a,b).$\\

\paragraph{Division Algorithm:} Given integers $a > b$, there exists unique integers $k, r$ such that 

\[a=kb+r, 0\leq r < b\]\\

\paragraph{Euclidean Algorithm:} Given $a,b$, want output as $\gcd(a,b),m,n$ such that $am+bn=\gcd(a,b)$.

Set $a_1:=a,b_1:=b$. Use division to find $k_1, r_1$, such that

\[a_1=k_1b_1+r_1, (0\leq r_1< b_1)\]

Now set $a_2:=b_1, b_2:=r_1$. Find $k_2, r_2$, such that 

\[a_2 = k_2b_2+r_2, (0\leq r_2< b_2)\]

Set $a_3:=b_2, b_3:=r_2$

$\cdots$

Eventually,

\[a_n=k_nb_n\]

Claim: when reminder is gone, we stop, and $\gcd(a,b)=b_n=r_{n-1}.$
\\

Example: $a_1=343, b_1=154$

\[
\begin{split}
	343&=2\times 154+ 35\\
	154&=4\times 35 + 14\\
	35&=2\times 14 + 7\\
	14&=2\times 7
\end{split}
\]
So $\gcd(343,154)=7.$\\

Observe $\gcd(a_n,b_n)=b_n$. So it's enough to show $\gcd(a_i,b_i)=\gcd(a_{i+1}, b_{i+1})$, for each $i=1,\dots, n-1.$ ($\Rightarrow \gcd(a_1,b_1)=\gcd(a_2,b_2)=\cdots = \gcd(a_n,b_n)=b_n).$\\

\[a_i=k_ib_i+r_i =k_ia_{i+1}+b_{i+1}\]

We are gonna prove $\gcd(a_i,b_i)\leq\gcd(a_{i+1},b_{i+1})$ and backward $\gcd(a_{i+1},b_{i+1})\geq\gcd(a_i,b_i)$.

\begin{itemize}
	\item $\gcd(a_i,b_i)|r_i=b_{i+1}$, also $\gcd(a_i,b_i)|a_{i+1}=b_i$, therefore $\gcd(a_i, b_i)\leq \gcd(a_{i+1}, b_{i+1})$
	\item $\gcd(a_{i+1}, b_{i+1})|a_i$, also $\gcd(a_{i+1}, b_{i+1})|b_i=a_{i+1}$, therefore $\gcd(a_{i+1},b_{i+1})\geq\gcd(a_i,b_i)$
\end{itemize}

Done.\\

Back to the example:

\[
\begin{split}
	35&=343-2\times 154\\
	14&=154-4\times 35\\
	7&=35-2\times 14\\
	&=9\times343-20\times 154\\
\end{split}
\]

\begin{enumerate}
	\item Why is Number Theory hard? $\mathbb{Z}$ is not a "field" (cannot divide)
	\item Think how fast this algorithm.
\end{enumerate}

\end{document}