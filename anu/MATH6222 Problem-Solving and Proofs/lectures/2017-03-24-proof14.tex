\documentclass[a4paper, 11pt, twoside]{article}
\usepackage{amssymb}
\usepackage{amsmath}
\begin{document}
\title{MATH6222 week 5 lecture 14}
\author{Rui Qiu}
\date{2017-03-24}

\maketitle

Last time we have:

\[{n \choose k} = {n-1 \choose k-1} + {n-1 \choose k}\]\\

Claim that ${n \choose k} := $ coefficient of $x^ky^{n-k}$ in $(x+y)^n$.

The reasoning is: pick $k$ number of $x$'s from $n$ number of $(x+y)$'s, and automatically pick $n-k$ number of $y$'s from $n$ number of $(x+y)$'s.\\

\paragraph{Pascal's Formula:} \[{n\choose k} = {n-1 \choose k} + {n-1 \choose k-1}\]

\paragraph{Proof:}\ \\

\begin{enumerate}
	\item Formula (direct proof)
	\[\frac{n!}{k!(n-k)!}=\frac{(n-1)!}{(n-1)!(n-1-k)!}+\frac{(n-1)!}{(n-1)!(n-k)!}\]
	\item Recall ${n \choose k}$ is the number of bug paths to the coordinate $(n-k, k)$. Interpret it.
	\item $(x+y)^n = \sum\limits^n_{k=0} {n \choose k} x^ky^{n-k} = y^n + nxy^{n-1} + {n \choose 2}x^2y^{n-2} + \cdots$\\
	Recall $(x+y)^n = (x+y)(x+y)^{n-1}$ where $(x+y)^{n-1}$ again is equal to $\sum\limits^{n-1}_{k=0}{n-1 \choose k}x^ky^{n-1-k}$\\
	\[\sum\limits^{n-1}_{k=0}{n-1 \choose k}x^{k+1}y^{n-1-k}+ \sum\limits^{n-1}_{k=0}{n-1 \choose k}x^ky^{n-k}\]
	We claim that the first term is equal to:
	\[\sum\limits^{n-1}_{k=0}{n-1 \choose k}x^{k+1}y^{n-1-k} = \sum\limits^n_{k=1}{n-1 \choose k-1}x^ky^{n-k}\]
	So the original formula becomes:
	\[
	\begin{split}
		\sum\limits^n_{k=1}{n-1 \choose k-1}x^ky^{n-k} + \sum\limits^{n-1}_{k=0}{n-1 \choose k}x^ky^{n-k} &= \sum\limits^n_{k=0}\bigg[{n-1 \choose k-1} + {n-1 \choose k}\bigg]x^ky^{n-k}\\
	\end{split}
	\]
\end{enumerate}

Define ${n \choose k}=0$ if $n,k$ do not satisfy $0\leq k\leq n.$\\

\paragraph{Probability:} Given $n$ equally likely outcomes, an event is defined to be a subset of these outcomes, and the probability of an event is just by definition $\frac{|A|}{n}.$\\

What is the probability of being dealt (for 5 cards):

\begin{itemize}
	\item a pair
	\item a straight
	\item a flush
\end{itemize}

We have ${52 \choose 5}$ possible hands. (in fact $2598960$)\\

To get a pair, we make following choices:
\begin{itemize}
	\item choose a rank for which we have two cards: $13$.
	\item choose two suits for the pair: ${4 \choose 2}$.
	\item choose three ranks from my remaining set of $12$ ranks: ${12 \choose 3}$.
	\item choose a suit three times: $4^3$.
\end{itemize}

Multiply them together, about $1098240$, so probability is around $42\%$.\\

To get a straight,
\begin{itemize}
	\item choose the lowest rank: $9$ (consider A, 2, 3, 4, 5 as a non-straight)
	\item choose one of $4$ suits $5$ times: $4^5$.
	\item eliminate the number of straight flushes $4$.
\end{itemize}

$9\times(4^5-4)$, probability around $0.35\%$.

\end{document}