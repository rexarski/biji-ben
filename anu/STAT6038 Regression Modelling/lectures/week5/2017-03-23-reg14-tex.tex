\documentclass[a4paper, 11pt, twoside]{article}
\usepackage{amssymb}
\usepackage{amsmath}
\begin{document}
\title{STAT6038 week 5 lecture 14}
\author{Rui Qiu}
\date{2017-03-23}

\maketitle

\paragraph{Assessing the underlying (model-specific) assumptions.}

\[\epsilon_i \overset{iid}\sim N(0, \sigma^2)\]

\begin{enumerate}
	\item iid = independent and identically distributed
	\item $N$ = normally distributed errors
	\item mean of distribution is $0$ (guaranteed by the least squares estimation -> not really an assumption)
	\item constant variance $\sigma^2$ (homoscedasticity or homoskedasticity)
\end{enumerate}

We assess these assumptions using the residuals (observed errors)

\[e_i=Y_i-\hat{Y_i}, i = 1, 2, \dots , n\]

and we do this assessment using residual plots.\\

\textbf{Key assumptions (in order of importance)}

\begin{enumerate}
	\item errors are independent (no obvious problem)
	\item errors are identically distributed with constant variance $\sigma^2$ (homoscedastic errors)
	\item errors are normally distributed
\end{enumerate}

Use resident plots:

1 and 2 are best assessed using a \textbf{plot} of the (standardized) residuals vs. fitted values aka residual plot.

3 is test assessed using a normal quantile plot (qq plot)

Other plots may be useful in diagnosing (getting more details on ) problems observed in the main residual plot (and occasionally in normal qq plot).\\

 If residual plot has a "curvature" -- a definite pattern $\implies$ indicating dependence in the errors $\implies$ errors are not independent $\implies$ model is probably not appropriate.
 
 If residual plot shows a "heteroscedasticity" $\implies$ non-constant variance.
 
 If outliers... outliers...
 
\begin{itemize}
	\item lack of independence
	\item nor constant variance
	\item potential outlier
\end{itemize}



\end{document}